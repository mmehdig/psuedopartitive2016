\documentclass{beamer}
\usepackage{ stmaryrd }
\usetheme{Goettingen}

\title{Stratified reference: measurement}
\subtitle{Implementation of Champollion (2015)}

\author{Mehdi Ghanimifard}
\institute % (optional, but mostly needed)
{University of Gothenburg}

\date{Semantics and Pragmatics, 2016}

\subject{Linguistics}
\AtBeginSubsection[]
{
  \begin{frame}<beamer>{Outline}
    \tableofcontents[currentsection,currentsubsection]
  \end{frame}
}

\begin{document}

\begin{frame}
  \titlepage
\end{frame}

\begin{frame}{Outline}
  \tableofcontents
\end{frame}

\section{Measurement Puzzle}

\subsection{The puzzle}

\begin{frame}{The puzzle}
Why does pseudopartitives reject certain measure functions? 
  \begin{itemize}
  \item {
    five liters of water (volume)
  }
  \item {
    *five degrees Celsius of water (*temperature)
  }
  \end{itemize}
\end{frame}

\subsection{Previous work}

\begin{frame}{Previous work}{Monotonicity}

Schwarzschild (2006): Only \alert{monotonic} measure functions are admissible.

  \begin{itemize}
  \item {
    A measure function $\mu$ is monotonic iff for any two entities $a$ and $b$, if $a$ is a proper part of $b$, then $\mu(a) < \mu(b)$.
    \pause 
  }
  \item {   
    For example, volume is monotonic, but temperature is not monotonic.
  }
  \end{itemize}
\end{frame}

\begin{frame}{Previous work: Problem?}{Monotonicity}

  \begin{itemize}
  \item {   
    Five feet of snow fell on Berlin
  }
  \item {   
    The height for snow is not monotonic, otherwise the snow in West Berlin should have lower height compering to whole snow in Berlin.
  }
  \item {   
    Schwarzschild response: $a$ should be a proper ``Pragmatic Part" of $b$.
  }
  \end{itemize}
  
\end{frame}


\subsection{Novel observation}

\begin{frame}{Novel observation}{for-adverbials}

  \begin{itemize}
  \item {
    The same measurement functions rejected by psudopartitives are rejected by for-adverbials.
  }
  \item {
    five hours of driving (duration)
  }
  \item {
    *five miles per hour of driving (*speed)
  }
  \end{itemize}
  Strategy: Solve the aspect puzzle and the measurement puzzle in distributivity by introducing the \alert{stratified reference}.
\end{frame}

\section{Solution}

\subsection{For-adverbials}

\begin{frame}{For-adverbials}{Stratified reference presupposition}
  \begin{itemize}
  \item {
    For-adverbials presuppose \alert{stratified reference}.
  }
  \item {
    eat apples for three hours
    \[
    \forall e [
        \llbracket eat\ apple \rrbracket(e) \rightarrow
        e \in\ *\lambda e' \left\Big(
            \begin{tabular}{l}
            \llbracket eat\ apple \rrbracket(e') \land \\
            \varepsilon(\lambda t[hour(t)=3])(runtime(e'))
            \end{tabular}
        \right\Big)
    ]
    \]
  }
  \item {
    *eat ten apples for three hours
    \[
    \forall e [
        \llbracket eat\ ten\ apples \rrbracket(e) \rightarrow
        e \in\ *\lambda e'    \left\Big(
            \begin{tabular}{l}
                \llbracket eat\ ten\ apples \rrbracket(e') \land \\
                \varepsilon(\lambda t[hour(t)=3])(runtime(e'))
            \end{tabular}
        \right\Big)
    ]
    \]
    (Every eating-ten-apples event consists of subevents of eating-ten-apples with runtime of shorter than 3 hours)
  }
  \end{itemize}
\end{frame}

\subsection{Stratified reference}

\begin{frame}{Stratified reference}{Definition}
  \begin{itemize}
  \item {
    Definition:
    \[
    SR_{f,\varepsilon(K)}(P) = \forall x[P(x) \rightarrow x \in\ ^*\lambda y(P(y) \land \varepsilon(K)(f(y)))]
    \]
    (Any $x$ with property $P$ consists of parts who have property $P$ and these parts are granular in $f$ dimension with the scale of $K$) 
  }
  \item{
    $\varepsilon(K)(f(y))$ roughly means that the $f$-dimension of $y$ is small in scale of $K$.
  }
  \item{
    Intuitively, $x \in\ *\lambda y. P(y)$ means that $x$ consists of one or more parts such that $P$ holds for every part.
  }
  \end{itemize}
\end{frame}

\subsection{Pseudopartitives}

\begin{frame}{Pseudopartitives}{Parallels between for-adverbials and pseudopartitives}
  \begin{itemize}
  \item[1.] {
    Both reject if it fails to apply to parts of entity or event.
    \[
    SR_{f,\varepsilon(K)}(P) = \forall x[P(x) \rightarrow x \in\ ^*\lambda y(\color{red}P(y) \color{black} \land \varepsilon(K)(f(y)))]
    \]
    *five pounds of book. (with ``book" as a count noun)
    \pause
  }
  \item[2.] {
    Both reject if the value of measure function stays constant acorss the parts of entity or event.
    \[
    SR_{f,\varepsilon(K)}(P) = \forall x[P(x) \rightarrow x \in\ ^*\lambda y(P(y) \land \color{red}\varepsilon(K)(f(y))\color{black})]
    \]
    *five degrees Celsius of the water in the bottle.
  }
  \end{itemize}
\end{frame}

\begin{frame}{Pseudopartitives}{Snow example}
  \begin{itemize}
  \item {
    five feet snow
    \[
    SR_{height,\varepsilon(\lambda l.[feet(l)=5])}(\lambda y.snow(y)) = \\
    \forall x [
        snow(x) \rightarrow
        x \in\ *\lambda y \left\Big(
            \begin{tabular}{l}
                snow(y) \land \\
                \varepsilon(\lambda t[feet(t)=5])(height(y))
            \end{tabular}
        \right\Big)
    ]
    \]
  }
  \end{itemize}
\end{frame}

\begin{frame}{Pseudopartitives}{Mass vs count nouns}
  \begin{itemize}
  \item {
    100 grams of apple
    \[
    SR_{weight,\varepsilon(\lambda l.[gram(l)=100])}(\lambda y.apple(y)) = \\
    \forall x [
        apple(x) \rightarrow
        x \in\ *\lambda y \left\Big(
            \begin{tabular}{l}
                apple(y) \land \\
                \varepsilon(\lambda t[gram(t)=100])(weight(y))
            \end{tabular}
        \right\Big)
    ]
    \]
  }
  \item {
    *100 grams of apples
    \[
    SR_{weight,\varepsilon(\lambda l.[gram(l)=100])}(\lambda y.*apple(y)) = \\
    \forall x [
        *apple(x) \rightarrow
        x \in\ *\lambda y \left\Big(
            \begin{tabular}{l}
                *apple(y) \land \\
                \varepsilon(\lambda t[gram(t)=100])(weight(y))
            \end{tabular}
        \right\Big)
    ]
    \]
  }  \end{itemize}
\end{frame}

\section{Implementation}
\subsection{Presupposition}

\begin{frame}{Presupposition}{Partial function}

Partial function are used to represent presuppositions. The representation of partial function in Champollion (2015):

\[
\lambda x:\phi.\psi
\]

If $\phi$ holds it returns $\psi$ otherwise undefined.

\begin{itemize}
    \item{My solution:
    \[
    \lambda x.\partial(\phi)(\psi)
    \]

    In Lambda Calculator:
    
    \[
    \lambda x.partial(phi)(psi)
    \]
    }
\end{itemize}
\end{frame}

\subsection{For-adverbials}

\begin{frame}{For-adverbials}{For-adverbials for entities and events}

Based on Champollion (2015) $for$ is represented as follows:
    \[
    \llbracket for \rrbracket = \lambda \tau_{\langle v,i \rangle} \lambda M_{\langle i,t \rangle} \lambda P_{\langle v,t \rangle} \lambda e.\partial(SR_{\tau,\varepsilon(M)}(P))(P(e)\land M(\tau(e)))
    \]
    
    For example: ``he walked [for five miles]_{AdvP}":
    
    \[
    \llbracket for\ five\ miles \rrbracket 
    = \lambda P_{\langle v,t \rangle} \lambda e.\partial(SR_{\sigma,\varepsilon(\lambda l.[mile(l)=5])}(P))(P(e)\land [mile(\sigma(e))=5])
    \] 
    ($\sigma$ is the parameter for spatial extend, instead of runtime.)

    \[
    \llbracket walk\ for\ five\ miles \rrbracket 
    = \\
    \lambda e.\partial(SR_{\sigma,\varepsilon(\lambda l.[mile(l)=5])}(\lambda e'.walk(e'))(walk(e)\land [mile(\sigma(e))=5])
    \] 

\end{frame}

\begin{frame}{For-adverbials}{For-adverbials for entities and events}

On entities, for example ``five miles of railroad tracks":

    \[
    \llbracket (for) \rrbracket = \lambda f_{\langle e,i \rangle} \lambda K_{\langle i,t \rangle} \lambda P_{\langle x,t \rangle} \lambda x.\partial(SR_{f,\varepsilon(K)}(P))(P(x)\land K(f(x)))
    \]
    
    \[
    \Rightarrow \llbracket (for)\ five\ miles \rrbracket 
    = \\
    \lambda P_{\langle e,t \rangle} \lambda x.\partial(SR_{\sigma,\varepsilon(\lambda l.[mile(l)=5])}(P))(P(x)\land [mile(\sigma(x))=5])
    \] 
    

    \[
    \Rightarrow \llbracket [(for)\ five\ miles]\ of\ railroad\ tracks \rrbracket 
    = \\
    \lambda x.\partial(SR_{\sigma,\varepsilon(\lambda l.[mile(l)=5])}(\lambda y.railroad(y)))(railroad(x)\land [mile(\sigma(x))=5])
    \] 

\end{frame}


\subsection{Lambda Calculator}

\begin{frame}{Lambda Calculator}{Implementation}

\begin{itemize}
    \item{In our application, the partial function is only for truth-values:
    
    $constants\ of\ type\ <t,<t,t>> : partial$
    }
    \item{Granularity function:
    
    $constants\ of\ type\ <it,it> : eps$
    }
    \item{Stratified reference parametrized constant function can be used in presupposition:

    $constants\ of\ type\ <ei,<it,<et,t>>> : SR$ \\
    
    $\Rightarrow SR(f_{ei})(\varepsilon(K_{it}))(P_{et})$ \\
    
    $constants\ of\ type\ <vi,<it,<vt,t>>> : SRv$

    $\Rightarrow SRv(T_{vi})(\varepsilon(M_{it}))(Q_{vt})$ \\
    
    }
\end{itemize}
\end{frame}

\begin{frame}{Thank you}
    Thank you
\end{frame}

\end{document}


